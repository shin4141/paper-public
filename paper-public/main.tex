\documentclass[10pt,conference]{IEEEtran} % 変更可: \documentclass[sigconf]{acmart}

% ====== Packages (最小) ======
\usepackage[utf8]{inputenc}
\usepackage{graphicx}
\usepackage{booktabs}
\usepackage{amsmath}
\usepackage[hidelinks]{hyperref}

% ====== Metadata ======
\title{Decision-OS V5 (SiriusA): A Human-in-the-Loop Decision System with Aspire-First Layer}
\author{
  \IEEEauthorblockN{Shin (BSC)}
  \IEEEauthorblockA{Decision-OS Lab\\
  \texttt{https://github.com/shin4141/paper-public}}
}

\begin{document}
\maketitle

\begin{abstract}
% 150–250 words を目安。とりあえず abstract.txt の内容を後で貼り替える想定。
This paper introduces Decision-OS V5 (SiriusA), a human-in-the-loop decision framework that
prioritizes aspiration (Aspire-first) while preserving guardrails such as EV gates, WAIT48h,
and intervention rules. We report a minimal public set of principles, diagrams (sans thresholds),
and anonymized KPIs for reproducible evaluation.
\end{abstract}

\section{Introduction}
% 目的/課題/貢献(Contributions)を3–5段落

\section{Background \& Lineage}
% Polaris → DGIS → SiriusA の系譜(数値や内部プロンプトの全文は出さない)

\section{System Overview}
% フロー図(figures/)参照。しきい値の実数や検知語の全文は非公開。

\section{Evaluation Protocol}
% 評価設計のみ。匿名KPI (Adherence, Decision Time, FPR/FNR, Net Benefit, Brier/ECE)

\section{Ethics \& Safety}
% 非医療・2タップ手動・HITL・個情無し・保持/破棄・誤発報対策など

\section{Limitations \& Future Work}

\bibliographystyle{IEEEtran}
\bibliography{refs}

\end{document}
